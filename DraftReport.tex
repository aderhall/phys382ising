\RequirePackage[2020-02-02]{latexrelease}
\documentclass[letter,scriptaddress,twocolumn, prl]{revtex4}

\usepackage{amsmath}%,amssymb} 
\usepackage{makeidx}
\usepackage{amsfonts}
\usepackage[ansinew]{inputenc}
%\usepackage[usenames,dvipsnames]{pstricks}
%\usepackage{subfigure}
\usepackage{epsfig}
%\usepackage{pst-grad} % For gradients
%\usepackage{pst-plot} % For axes
\usepackage[colorlinks,hyperindex]{hyperref}
\hypersetup
{
colorlinks,%
citecolor=black,%
linkcolor=black,%
urlcolor=black,%
}

\setlength\textheight{24.5cm}

% --- Comandos novos ---
\newcommand{\dket}[1]{\left| #1 \right)}
\newcommand{\E}[1]{\frac{\hbar^2 #1 ^2}{2m_0}}
\newcommand{\dbra}[1]{\left( #1 \right|}
\newcommand{\dsubmin}[1]{\left( #1 \right)}
\newcommand{\dbraket}[2]{\left( #1 | #2 \right)}
\newcommand{\dbraketm}[3]{\left( #1 \left| #2 \right| #3 \right)}
\newcommand{\ket}[1]{\left| #1 \right\rangle}
\newcommand{\bra}[1]{\left\langle #1 \right|}
\newcommand{\submin}[1]{\left\langle #1 \right\rangle}
\newcommand{\braket}[2]{\left\langle #1 \right. \left| #2 \right\rangle}
\newcommand{\braketm}[3]{\langle #1 \mid #2 \mid #3 \rangle}
\newcommand{\pinterno}[2]{\left( #1 , #2 \right)}
\newcommand{\comut}[2]{\left[ #1 , #2 \right]} % THE COMUTATOR
\newcommand{\seitz}[2]{\left\{ \, #1 \mid  #2 \, \right\}}
\newcommand{\rep}{\emph{rep} }
\newcommand{\irep}{\emph{irrep} }
\newcommand{\ordem}[1]{\mid #1 \mid}
\newcommand{\op}[1]{\mathbb #1 }
\newcommand{\group}[1]{\mathcal #1 }
\newcommand{\vet}[1]{\mathbf #1 }
\newcommand{\argu}[1]{\left( #1 \right)}
\newcommand{\kp}{\vet{k}\cdot\vet{p}}

\makeindex

%--------------------------------------------------------
\begin{document}

\title{A brief and concise description of RevTeX 4 package}

\author{Alex Roseman}
\author{Adrian Hall}
\date{today}

\begin{abstract}

\end{abstract}

\maketitle

\section{Introduction}

\begin{equation}
	\label{eq:test}
	\vec{f} = m \vec{a} .
\end{equation}
So, equation \eqref{eq:test} or \ref{eq:test} must be solved for position function so one can know everything about the system. But how to solve this. Here, $\vec{f}$ is a sum over all the forces acting on this particle, $\sum\limits_{\forall i} f_i$.
%In my books.bib, I have an entry named goldstein80 wich refers to Classical Mechanics by Goldstein. The same for greene1995.

% Fig.\,\ref{fig:example}.

%\begin{figure}[htbp]
%	\begin{center}
%		\includegraphics[width=0.3\textwidth]{exampleFig} %depending on the latex compiler, you can omit the file extension
%		\caption{Include your caption here.}
%		\label{fig:example}
%	\end{center}
%\end{figure}

\section{Acknowledgements}
	Prof. Navon, Prof. Newburgh, HongJoon

%\bibliographystyle{unsrt} 
%\bibliography{/home/thiago/bibtex/articles,/home/thiago/bibtex/books}

\begin{thebibliography}{}
	
	\bibitem{mcdermott}
	Y.-F.~Chen {\it et al.},
	%``Microwave Photon Counter Based on Josephson Junctions,''
	Phys.\ Rev.\ Lett.\  {\bf 107}, 217401 (2011).
	
	
\end{thebibliography}

\end{document}
